\documentclass[a4paper,11pt]{report}

\begin{document}

\section{Introduction}
\label{introduction}

The paper proposes an communicating-automata-based model to formally
verify of single page and multi page web applications. The authors
propose an ``In'\,' operator for LTL in order to simplify LTL. This
doesn't add expressiveness, but makes LTL formulas in this specific
domain more succint.

The proposed approach is dynamic and black-box, also known as run-time
verification. This is so that the approach is not dependant on the
source code being accessible, and assumes only the request-response flow
is observable.

\section{Automata-based modeling of web applications}
\label{automata-based-modeling-of-web-applications}

Disclaimer: this is not an exaustive testing like traditional model
checking, but should be considered as ``passive testing'\,'.

\subsection{Modeling approach}
\label{modeling-approach}

\subsection{Single window browsing}
\label{single-window-browsing}

\subsection{Multiple window browsing}
\label{multiple-window-browsing}

\section{LTL and the ``In'' operator}
\label{ltl-and-the-in-operator}

In order to represent more succintly LTL formulas in the domain of web
applications, the authors extend the LTL syntax with the \textbf{In}
operator.

\section{Results}
\label{results}

\end{document}
