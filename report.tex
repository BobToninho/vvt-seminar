\documentclass[a4paper,11pt]{article}

\title{Report: A formal approach for run-time verification of web applications using scope-extended LTL}
\author{Roberto Tonino}

% % cool font {{{
% \usepackage[utf8]{inputenc}
\usepackage{libertine}
% \usepackage{libertinust1math}
% \usepackage[T1]{fontenc}
% % }}}

% biblatex {{{
\usepackage{biblatex}
\addbibresource{report.bib}
% }}}

\usepackage{hyperref}
\begin{document}

\maketitle
\tableofcontents

\section{Introduction}
\label{introduction}

In this short report it is summarized the paper \citetitle{Haydar2013}.

The paper proposes a communicating-automata-based model to formally
verify single page and multi page web applications. The authors
propose an ``In'' operator for LTL in order to simplify LTL. This
doesn't add expressiveness, but makes LTL formulas in this specific
domain more succint.

The proposed approach is dynamic and black-box, also known as run-time
verification. This is so that the approach is not dependant on the
source code being accessible, and assumes only the request-response flow
is observable.

\section{Automata-based modeling of web applications}
\label{automata-based-modeling-of-web-applications}

Disclaimer: this is not an exaustive testing like traditional model
checking, but should be considered as ``passive testing''.

\subsection{Modeling approach}
\label{modeling-approach}

The monitoring approach proposed contains three main components or modules---monitoring module, analysis module and model checking module. The monitoring module intercepts HTTP requests and responses of the WAUT. The analysis module generates a Promela model taking as input the intercepted traces. Finally, the model checking module verifies user-defined properties against the model generated by the analysis module and produces a counterexample. It uses the Spin model checker.

\subsection{Single window browsing}
\label{single-window-browsing}

\subsection{Multiple window browsing}
\label{multiple-window-browsing}

\section{LTL and the ``In'' operator}
\label{ltl-and-the-in-operator}

In order to represent more succintly LTL formulas in the domain of web
applications, the authors extend the LTL syntax with the \textbf{In}
operator.

\section{Results}
\label{results}

\printbibliography

\end{document}

% vi: fdm=marker
