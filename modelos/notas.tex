% Opções de tamanho de fonte: Huge, huge, LARGE, Large, large, normalsize, small

% Exemplo de título e subtítulo:
\def \titulo {Título do Artigo\LARGE{: Subtítulo do Artigo}}

% Exemplo de dois autores:
\def \autor {Autor do Silva \& Autor Ferreira}

% Exemplo de dois autores com a mesma filiação e emails distintos:
\def \autor {
Autor~A\raisebox{5pt}{\normalsize$^{\ast 1}$} \&
Autor~B\raisebox{5pt}{\normalsize$^{\dag 1}$}\\
{\normalfont\footnotesize
$^1$ Redator, Instituto A\quad
$^\ast$\texttt{emailA@email.com}\quad
$^\dag$\texttt{emailB@email.com}}}
% Símbolos para notas de rodapé: \ast, \dag, \ddag, \S, \P, ...

% Exemplo de tradutor e revisor
% \def \tradutor {Fulano da Silva\newline\& \normalfont{Revisado por} \textsc{Beltrano da Silva}}

% -------------------------------------------------------------
% V. 1.2
% -------------------------------------------------------------
%
% 1.2:
%  seções, subseções e sub-subseções quebram a página automaticamente, exceto "as primeiras" (a primeira subseção, etc).
%  estrutura dos arquivos e main.tex modificados
%  tamanho da fonte para folha a4 modificada.
% 1.1.1:
%  fix de newline depois da capa
% 1.1.0:
%  novo feature: opção para gerar PDF para serem visualizados em uma tela de celular.
%  fix: Informações do artigo ficaram mais centralizadas verticalmente.
%  improv: main.tex mais simples
% -------------------------------------------------------------
